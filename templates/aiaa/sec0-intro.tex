\section{Introduction}
\label{sec:intro}

\lettrine{T}his document describes the things you need to set up to follow Paul's workflow for writing papers. Here's the software you need to get started:
\begin{enumerate}
\item LaTeX distribution that enables you to run \texttt{latex}, \texttt{pdflatex}, \texttt{bibtex}, \texttt{dvips}, and \texttt{ps2pdf} from the command line,
\item text editor (I use Emacs),
\item Bibtex reference manager (I use JabRef (\url{jabref.org}) because it's based on Java so it's cross platform),
\item Dropbox for sharing a folder.
\end{enumerate}
There are certainly plenty of ways to share LaTeX documents and papers. This is just what's worked for me. I have a directory structure for a paper with the following directories:
\begin{itemize}
\item \texttt{paper\_v0}, where the paper tex files are along with a \texttt{figs} directory; if big changes happen, I might make a new \texttt{paper\_vX} folder
\item \texttt{references}, where I keep PDFs of papers I'm referencing
\item \texttt{code}, where I keep scripts for numerical experiments (helps reproducibility and recreating figures)
\item \texttt{arxiv}, I make a separate folder for the version I put on arXiv
\item \texttt{reviewer\_response}, where I keep a text file of the initial reviews and a LaTeX file with reviewer responses
\item \texttt{latexdiff}, when I submit a revision I include the results of \texttt{latexdiff} (see example syntax in folder)
\item \texttt{paper\_r0}, a folder for the revision; this usually begins as a duplicate of \texttt{paper\_v0}
\end{itemize}