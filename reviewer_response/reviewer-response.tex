%%This is a very basic article template.
%%There is just one section and two subsections.

\documentclass[11pt]{article}

\usepackage{amsmath,amsfonts}
\usepackage[letterpaper]{geometry}
\usepackage{helvet}
%\usepackage{newproof}
%\usepackage{amsthm}
\usepackage{ntheorem}
\theoremheaderfont{\sffamily\scshape}
\newtheorem{comment}{comment}[section]
\theoremstyle{nonumberplain}
\theoremheaderfont{\sffamily\scshape}
\theorembodyfont{\normalfont}
\newtheorem{response}{response}
\theoremindent 2em
\newtheorem{changes}{changes}
\input{pauls-macros}

%\renewcommand{\qedsymbol}{}

\title{Response to Referees of 'TITLE'}
\author{Paul G.~Constantine \and Coauthors}
\begin{document}
\maketitle

We have revised the manuscript in accordance with the and Reviewers' comments for improvements. In what follows, we address each comment individually. For reference, we have included the results of \texttt{latexdiff} (without math-markup) applied to the revised manuscript. We suggest that the reviewer use this document to highlight only general changes, since the output from the script does not capture each difference perfectly.

\section{Reviewer \#1}

\begin{comment} 
\end{comment}
\begin{response}
\end{response}
\begin{changes}
\end{changes}

\section{Reviewer \#2}

\begin{comment} 
\end{comment}
\begin{response}
\end{response}
\begin{changes}
\end{changes}


\end{document}
